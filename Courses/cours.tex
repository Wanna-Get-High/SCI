\documentclass[a4paper,10pt]{article}
\input{/Users/WannaGetHigh/workspace/latex/macros.tex}

\title{Semaine 1 : Cour RVI}
\author{Fran�ois \bsc{Lepan}}

\begin{document}
\maketitle

\section{Definition RVI}

Les simulateurs sont les anc�tres des dispositifs de RV et les moteurs ...

Un dispositifs de RV est un ensemble de mat�riels et de logiciels permettant d'immerger l'usager dans un mode virtuel en lui proposant des possibilit� d'interaction.


\section{R�allit� augment�}

Superpos� R�el et Virtuel.
Id�alement sur le "site" r�el.

\section{Simulateur Automobile : Auto-�cole}
Il faut :
\begin{itemize}
\item poste de conduite (habitacle, si�ge volant) + connections de p�riph�riques
\item logiciel de simulation 3D (cartes, voiture, simulation comportement autres conducteurs)
\item des �crans + son
\item plateforme 6 axes -> les 6 degr�e de libert�
\end{itemize}

\section{Homme au centre du dispositif}

\begin{paragraph}{Vue}
\begin{tabular}{|c|c|}
\hline
Physiolgie & Techno disponible \\
\hline
Yeux & �cran+projecteur en salle \\
 & casques (�cran sur chaque oeuil)\\
 \hline
\end{tabular}
\end{paragraph}

\begin{paragraph}{Toucher}
\begin{tabular}{|c|c|}
\hline
Physiolgie & Techno disponible \\
\hline
d'abord les Mains &  \\
mais pas seulement & \\
\hline
\end{tabular}
\end{paragraph}

~\\

Le point difficile 

Les mains sont � la fois des outilles de perception mais surtout "d'action"

$\rightarrow$ toucher-tactile "capteurs sensitifs" 

$\rightarrow$ kinesth�siques "musculaire"

En plus on a un nombre de degr�s de libert�s consid�rable

$\rightarrow$ Les p�riph�riques � retour d'effort prennent en compte uniquement un point avec 3 degr� de libert�.

\begin{paragraph}{Ou�e}
\begin{tabular}{|c|c|}
\hline
Physiolgie & Techno disponible \\
\hline
Oreille & Casques, salles \\
& (entreprise Lille : A-Volute simulation de son) \\
\hline
\end{tabular}
\end{paragraph}

\begin{paragraph}{Odorat}
\begin{tabular}{|c|c|}
\hline
Physiolgie & Techno disponible\\
\hline
Nez &  plein de technologies (imprimantes, diffuseurs) \\
\hline
\end{tabular}
\end{paragraph}

\begin{paragraph}{Go�t}
\begin{tabular}{|c|c|}
\hline
Physiolgie & Techno disponible \\
\hline
Bouche & presque rien \\
\hline
\end{tabular}
\end{paragraph}

~\\

Il y a une forte interaction entre les sens !

\section{Le clavier}

$\rightarrow$ production de texte

$\rightarrow$ une touche oui/non par lettre (grand nombre de degr� de libert�)


\section{Interface d'action}
Se d�placer, d�couvrir

~	~$\rightarrow$ V�hicules

~	~$\rightarrow$ technos �normes (salle avec des dalles qui bouges, escaliers � 3 marches, OMNI)


Interaction

~	~$\rightarrow$ agir dans l'environnement

~	~$\rightarrow$ relation avec les autres (�tre � plusieurs dans le m�me monde virtuel)

$\rightarrow$ la vrai �volution actuelle : les capteurs

~	~$\rightarrow$ Wii

~	~$\rightarrow$ Kinect

~	~$\rightarrow$ Accel�rom�tre

~	~$\rightarrow$ GPS


\section{� voir}

le cinqui�mes sommeil B. Auxi�tre

Eidolon

\end{document}